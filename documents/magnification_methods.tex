\documentclass[12pt]{article}

\usepackage{amssymb}
\usepackage{geometry}
\geometry{textwidth=520pt,top=20mm,bottom=20mm}

\newcommand\MM{{\tt MulensModel}}


\begin{document} % ########################################

\begin{center}
{\LARGE Methods to calculate magnification in \MM}\\
\bigskip
Radek Poleski\\
last update: Jan 2022
\end{center}

\bigskip\bigskip

\MM offers a wide range of methods used to calculate magnifications. 
These methods are passed to {\tt Model} class using 
{\tt set\_magnification\_methods()} function.  For each method one has to 
pass the time ranges when the method will be used.  These parameters are 
passed in a list, e.g.:

\begin{verbatim}
model = Model(...)
model.set_magnification_methods(
    [2455745., 'Hexadecapole', 2455746., 'VBBL', 2455747., 'Hexadecapole', 2455748.])
\end{verbatim}

There are two other useful functions. 
First, {\tt set\_default\_magnification\_method()}\linebreak allows setting method that 
is used outside time ranges specified above.  
Second, \linebreak{\tt set\_magnification\_methods\_parameters()} allows 
providing additional parameters for calculations.  Currently, only 
{\tt VBBL} and {\tt Adaptive\_Contouring} allow providing these parameters. 

\bigskip\bigskip
Point lens methods:
\begin{itemize}
\item {\tt point\_source} -- the simplest thing that exists, also called ``Paczy\'nski curve'':\linebreak $A(u) = \left(u^2+2\right)/\left(u\sqrt{u^2+4}\right)$.
\item {\tt finite\_source\_uniform\_Gould94} -- for the finite source with uniform profile (i.e., no limb-darkening effect) use approximation presented by Gould (1994).  It works only for small $\rho$, i.e, $\rho\lesssim0.1$.
\item {\tt finite\_source\_uniform\_WittMao94} -- for the finite source with uniform profile use method presented by Witt and Mao (1994).  It interpolates pre-computed tables.
\item {\tt finite\_source\_LD\_WittMao94} -- for the finite source with limb darkening integrate many uniform profiles. For each uniform profile, {\tt finite\_source\_uniform\_WittMao94} method is used. For description on how the uniform profiles are combined see, e.g., Bozza et al. (2018).
\item {\tt finite\_source\_LD\_Yoo04} -- for the finite source with limb darkening use Yoo et al.~(2004) approximation.  It works only for small $\rho$, i.e, $\rho\lesssim0.1$.
\item {\tt finite\_source\_uniform\_Lee09} -- for the finite source with uniform profile (Lee et al.~2009) but works well for large $\rho$ as well (e.g., $\rho = 2$).  It is significantly slower than approximate method {\tt finite\_source\_uniform\_Gould94}.
\item {\tt finite\_source\_LD\_Lee09} -- for the finite source with limb darkening that works well for large sources (e.g., $\rho = 2$).  This method is much slower than {\tt finite\_source\_LD\_Yoo04}.
\end{itemize}
Please note that {\tt finite\_source\_uniform\_Gould94} and {\tt finite\_source\_LD\_Yoo04} interpolate pre-com\-pu\-ted values.  This interpolation should be very accurate.
If you want to test it, then request direct calculations using {\tt finite\_source\_uniform\_Gould94\_direct} or {\tt finite\_source\_LD\_Yoo04\_direct}.
For $u/\rho$ that is outside pre-computed values the direct calculation is used as well.

\bigskip\bigskip
Binary lens methods:
\begin{itemize}
\item {\tt point\_source} -- assumes the source is just a point (hence not valid near caustics) and solves fifth order complex polynomial once.
\item {\tt quadrupole} -- uses Taylor expansion -- evaluates point-source magnification at 9 points.  Works only outside caustic.
\item {\tt hexadecapole} -- uses Taylor expansion -- evaluates point-source magnification at 13 points.  Works only outside caustic.
\item {\tt VBBL} -- Bozza (2010) method -- finite source with limb darkening.  Most widely used method nowadays.  Parameters that ca be set: {\tt accuracy}.
\item {\tt Adaptive\_Contouring} -- Dominik (2007) method -- finite source with limb darkening.  Parameters that can be set: {\tt accuracy} and {\tt ld\_accuracy}.
\item {\tt point\_source\_point\_lens} -- approximates binary lens as a single lens.  It is useful when binary lens effects are negligible and binary lens calculations may cause numerical errors, e.g., $q\approx10^{-6}$ and source far from caustics.  
\end{itemize}
Note that if you define shear and convergence, then \MM uses properly modified versions of: {\tt point\_source}, {\tt quadrupole}, {\tt hexadecapole}, or {\tt VBBL}.

\bigskip\bigskip
Triple lens methods -- under construction.  Please come back later!


\end{document}

